\usemodule[json]
\startluacode

  userdata = userdata or {}
  local json = utilities.json

  userdata.show_recipe = function(recipe)

  local lua_recipe  = json.tolua(recipe).recipe
  local ingredients = lua_recipe.ingredients
  local cooking     = lua_recipe.cooking

  context.subject(lua_recipe.title)

  local show_value = function(value)
    context.NC() context(value)
    context.EQ() context(lua_recipe[value])
    context.NC() context.NR()
  end

  context.starttabulate()
    show_value("source")
    show_value("carbs")
    show_value("protein")
    show_value("cal")
  context.stoptabulate()

  context.subsubject("Ingredients")
  context.startitemize{"packed, intro"}
  for i = 1,#ingredients do
    context.startitem()
    context(ingredients[i].item)
    context.stopitem()
  end
  context.stopitemize()

  context.subsubject("Cooking")
  context.startenumerate{"packed, intro"}
  for i = 1,#cooking do
    context.startitem()
    context(cooking[i].step)
    context.stopitem()
  end
  context.stopitemize()

  end
\stopluacode
% Note that I use the braces around #1 to make the input
% syntax slightly simpler
\define[1]\Recipe
    {\ctxlua{userdata.show_recipe([==[{#1}]==])}}
\setuphead[subject][style=\bfb]
\setuphead[subsubject][style=\bfa]
\starttext

\Recipe { "recipe": {
          "title":"First recipe",
          "source":"My first cookbook",
          "carbs":"1 oz",
          "fat":"1 oz",
          "protein":"1 oz",
          "cal":"100 kcal",
          "ingredients": [
              {"item":"Eggs"},{"item":"Oil"},
              {"item":"Nuts"}
          ],
          "cooking": [
              {"step":"Mix eggs and oil"},
              {"step":"Add nuts"}
          ]
      }
    }

\stoptext
